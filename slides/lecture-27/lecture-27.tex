\documentclass{beamer}\usepackage[]{graphicx}\usepackage[]{color}
%% maxwidth is the original width if it is less than linewidth
%% otherwise use linewidth (to make sure the graphics do not exceed the margin)
\makeatletter
\def\maxwidth{ %
  \ifdim\Gin@nat@width>\linewidth
    \linewidth
  \else
    \Gin@nat@width
  \fi
}
\makeatother

\definecolor{fgcolor}{rgb}{1, 0.894, 0.769}
\newcommand{\hlnum}[1]{\textcolor[rgb]{0.824,0.412,0.118}{#1}}%
\newcommand{\hlstr}[1]{\textcolor[rgb]{1,0.894,0.71}{#1}}%
\newcommand{\hlcom}[1]{\textcolor[rgb]{0.824,0.706,0.549}{#1}}%
\newcommand{\hlopt}[1]{\textcolor[rgb]{1,0.894,0.769}{#1}}%
\newcommand{\hlstd}[1]{\textcolor[rgb]{1,0.894,0.769}{#1}}%
\newcommand{\hlkwa}[1]{\textcolor[rgb]{0.941,0.902,0.549}{#1}}%
\newcommand{\hlkwb}[1]{\textcolor[rgb]{0.804,0.776,0.451}{#1}}%
\newcommand{\hlkwc}[1]{\textcolor[rgb]{0.78,0.941,0.545}{#1}}%
\newcommand{\hlkwd}[1]{\textcolor[rgb]{1,0.78,0.769}{#1}}%
\let\hlipl\hlkwb

\usepackage{framed}
\makeatletter
\newenvironment{kframe}{%
 \def\at@end@of@kframe{}%
 \ifinner\ifhmode%
  \def\at@end@of@kframe{\end{minipage}}%
  \begin{minipage}{\columnwidth}%
 \fi\fi%
 \def\FrameCommand##1{\hskip\@totalleftmargin \hskip-\fboxsep
 \colorbox{shadecolor}{##1}\hskip-\fboxsep
     % There is no \\@totalrightmargin, so:
     \hskip-\linewidth \hskip-\@totalleftmargin \hskip\columnwidth}%
 \MakeFramed {\advance\hsize-\width
   \@totalleftmargin\z@ \linewidth\hsize
   \@setminipage}}%
 {\par\unskip\endMakeFramed%
 \at@end@of@kframe}
\makeatother

\definecolor{shadecolor}{rgb}{.97, .97, .97}
\definecolor{messagecolor}{rgb}{0, 0, 0}
\definecolor{warningcolor}{rgb}{1, 0, 1}
\definecolor{errorcolor}{rgb}{1, 0, 0}
\newenvironment{knitrout}{}{} % an empty environment to be redefined in TeX

\usepackage{alltt}
\usepackage{../371g-slides}
\title{Simulation 1}
\subtitle{Lecture 27}
\author{STA 371G}
\IfFileExists{upquote.sty}{\usepackage{upquote}}{}
\begin{document}
  
  

  \frame{\maketitle}

  % Show outline at beginning of each section
  \AtBeginSection[]{
    \begin{frame}<beamer>
      \tableofcontents[currentsection]
    \end{frame}
  }

  %%%%%%% Slides start here %%%%%%%

  \begin{darkframes}
    \begin{frame}{What is simulation?}
      \begin{itemize}[<+->]
        \item Sometimes (e.g., using decision trees), it's possible for us to work out the expected outcome for a situation analytically.
        \item But decision trees only tell us about averages (expected value), and given only one ``bite at the apple,'' our actual outcome may not be close to the average.
        \item And what do we do when faced with a scenario that can't be modeled by decision trees, e.g., predicting the value of a portfolio at retirement age?
        \item Simulation lets us get a picture of the full distribution of possible outcomes in just about any problem scenario.
        \item We'll start with some silly examples today and then delve into business applications next time.
      \end{itemize}
    \end{frame}

    \begin{frame}{Example 1: Coin flipping}
      Suppose we flip a coin 10 times. What will the distribution of the number of heads look like?
      \lc
    \end{frame}

    \begin{frame}[fragile]
      \fontsize{10}{10}\selectfont
\begin{knitrout}
\definecolor{shadecolor}{rgb}{0.137, 0.137, 0.137}\begin{kframe}
\begin{alltt}
\hlstd{results} \hlkwb{<-} \hlkwd{replicate}\hlstd{(}\hlnum{100000}\hlstd{, \{}
  \hlstd{flips} \hlkwb{<-} \hlkwd{sample}\hlstd{(}\hlkwd{c}\hlstd{(}\hlnum{0}\hlstd{,} \hlnum{1}\hlstd{),} \hlnum{10}\hlstd{,} \hlkwc{replace}\hlstd{=T)}
  \hlkwd{sum}\hlstd{(flips)}
\hlstd{\})}
\hlkwd{hist}\hlstd{(results,} \hlkwc{breaks}\hlstd{=}\hlnum{10}\hlstd{,} \hlkwc{col}\hlstd{=}\hlstr{'orange'}\hlstd{)}
\end{alltt}
\end{kframe}
\input{/tmp/figures/unnamed-chunk-2-1.tikz}

\end{knitrout}
    \end{frame}

    \begin{frame}[fragile]
      Theory tells us that the outcome of flipping a coin 10 times and counting heads should be a Binomial distribution with expected value $np=10\cdot 0.5=5$ and SD $\sqrt{np(1-p)}=\sqrt{10\cdot 0.5\cdot 0.5} \approx 1.58$.

      We can compare our simulated results against the theoretical results:
\begin{knitrout}
\definecolor{shadecolor}{rgb}{0.137, 0.137, 0.137}\begin{kframe}
\begin{alltt}
\hlkwd{mean}\hlstd{(results)}
\end{alltt}
\begin{verbatim}
[1] 5.00154
\end{verbatim}
\begin{alltt}
\hlkwd{sd}\hlstd{(results)}
\end{alltt}
\begin{verbatim}
[1] 1.581519
\end{verbatim}
\end{kframe}
\end{knitrout}
      \lc
    \end{frame}

    \begin{frame}
      \fullpagepicture{monty-hall}
    \end{frame}

    \begin{frame}
      \fullpagepicture{gameplay}
    \end{frame}

    \begin{frame}
      \fullpagepicture{car}
    \end{frame}

    \begin{frame}
      \fullpagepicture{goat}
    \end{frame}


    \begin{frame}{Example 3: Monty Hall and \emph{Let's Make a Deal}}
      \begin{itemize}[<+->]
        \item Monty Hall used to host the game show \emph{Let's Make a Deal}.
        \item There are three doors: two contain a goat, and one contains...\pause \alert{a new car}!
        \item You (the contestant) select one of three doors.
        \item Without revealing what is behind the selected door, Monty opens one of the other two doors to reveal a goat.
        \item Monty then gives you a choice: keep your original door, or switch to the other (unopened) door.
      \end{itemize}
    \end{frame}

    \begin{frame}{Example 2: Monty Hall and \emph{Let's Make a Deal}}
      Do you have a better chance of getting the car by switching, or by keeping your original selection---or does it not matter?
      \lc
    \end{frame}

    \begin{frame}[fragile]
      \fontsize{10}{10}\selectfont
\begin{knitrout}
\definecolor{shadecolor}{rgb}{0.137, 0.137, 0.137}\begin{kframe}
\begin{alltt}
\hlstd{doors} \hlkwb{<-} \hlkwd{c}\hlstd{(}\hlnum{1}\hlstd{,} \hlnum{2}\hlstd{,} \hlnum{3}\hlstd{)}
\hlstd{results} \hlkwb{<-} \hlkwd{replicate}\hlstd{(}\hlnum{100000}\hlstd{, \{}
  \hlcom{# Randomly select my door and the door with the car.}
  \hlstd{car.door} \hlkwb{<-} \hlkwd{sample}\hlstd{(doors,} \hlnum{1}\hlstd{)}
  \hlstd{my.door} \hlkwb{<-} \hlkwd{sample}\hlstd{(doors,} \hlnum{1}\hlstd{)}
  \hlcom{# If I chose the door with the car, he randomly opens one}
  \hlcom{#   of the other two doors.}
  \hlcom{# If I chose another door, he opens the remaining door.}
  \hlkwa{if} \hlstd{(car.door} \hlopt{==} \hlstd{my.door) \{}
    \hlstd{monty.opens.door} \hlkwb{<-} \hlkwd{sample}\hlstd{(}\hlkwd{setdiff}\hlstd{(doors, my.door),} \hlnum{1}\hlstd{)}
  \hlstd{\}} \hlkwa{else} \hlstd{\{}
    \hlstd{monty.opens.door} \hlkwb{<-} \hlkwd{setdiff}\hlstd{(doors,} \hlkwd{c}\hlstd{(my.door, car.door))}
  \hlstd{\}}
  \hlcom{# Switch doors: select the door that Monty did not open.}
  \hlstd{my.door} \hlkwb{<-} \hlkwd{setdiff}\hlstd{(doors,} \hlkwd{c}\hlstd{(monty.opens.door, my.door))}
  \hlstd{my.door} \hlopt{==} \hlstd{car.door}
\hlstd{\})}
\hlkwd{sum}\hlstd{(results)} \hlopt{/} \hlnum{100000}
\end{alltt}
\begin{verbatim}
[1] 0.66824
\end{verbatim}
\end{kframe}
\end{knitrout}
      \lc
    \end{frame}

    \begin{frame}[fragile]
      \fontsize{10}{10}\selectfont
\begin{knitrout}
\definecolor{shadecolor}{rgb}{0.137, 0.137, 0.137}\begin{kframe}
\begin{alltt}
\hlstd{doors} \hlkwb{<-} \hlkwd{c}\hlstd{(}\hlnum{1}\hlstd{,} \hlnum{2}\hlstd{,} \hlnum{3}\hlstd{)}
\hlstd{results} \hlkwb{<-} \hlkwd{replicate}\hlstd{(}\hlnum{100000}\hlstd{, \{}
  \hlcom{# Randomly select my door and the door with the car.}
  \hlstd{car.door} \hlkwb{<-} \hlkwd{sample}\hlstd{(doors,} \hlnum{1}\hlstd{)}
  \hlstd{my.door} \hlkwb{<-} \hlkwd{sample}\hlstd{(doors,} \hlnum{1}\hlstd{)}
  \hlcom{# If I chose the door with the car, he randomly opens one}
  \hlcom{#   of the other two doors.}
  \hlcom{# If I chose another door, he opens the remaining door.}
  \hlkwa{if} \hlstd{(car.door} \hlopt{==} \hlstd{my.door) \{}
    \hlstd{monty.opens.door} \hlkwb{<-} \hlkwd{sample}\hlstd{(}\hlkwd{setdiff}\hlstd{(doors, my.door),} \hlnum{1}\hlstd{)}
  \hlstd{\}} \hlkwa{else} \hlstd{\{}
    \hlstd{monty.opens.door} \hlkwb{<-} \hlkwd{setdiff}\hlstd{(doors,} \hlkwd{c}\hlstd{(my.door, car.door))}
  \hlstd{\}}
  \hlcom{# Don't switch; just check to see if I won the car.}
  \hlstd{my.door} \hlopt{==} \hlstd{car.door}
\hlstd{\})}
\hlkwd{sum}\hlstd{(results)} \hlopt{/} \hlnum{100000}
\end{alltt}
\begin{verbatim}
[1] 0.3306
\end{verbatim}
\end{kframe}
\end{knitrout}
    \end{frame}

    \begin{frame}[fragile]{Simulating random variables}
      We have seen how the \texttt{sample} command can be used to draw from a set of alternatives with equal probability (e.g., flipping a coin).

      The \texttt{rnorm} command can be used to draw randomly from a normal distribution. Let's create 10 random heights, with mean 68 (inches) and SD 4.

\begin{knitrout}
\definecolor{shadecolor}{rgb}{0.137, 0.137, 0.137}\begin{kframe}
\begin{alltt}
\hlkwd{rnorm}\hlstd{(}\hlnum{10}\hlstd{,} \hlnum{68}\hlstd{,} \hlnum{4}\hlstd{)}
\end{alltt}
\begin{verbatim}
 [1] 68.57607 63.99855 70.53150 65.56610 66.38570
 [6] 69.82433 67.64435 69.96917 72.92480 70.23245
\end{verbatim}
\end{kframe}
\end{knitrout}
      \lc
    \end{frame}

    \begin{frame}
      Let's check that \texttt{rnorm} works as advertised!

\begin{knitrout}
\definecolor{shadecolor}{rgb}{0.137, 0.137, 0.137}\begin{kframe}
\begin{alltt}
\hlkwd{hist}\hlstd{(}\hlkwd{rnorm}\hlstd{(}\hlnum{1000}\hlstd{),} \hlkwc{col}\hlstd{=}\hlstr{'orange'}\hlstd{)}
\end{alltt}
\end{kframe}
\input{/tmp/figures/unnamed-chunk-8-1.tikz}

\end{knitrout}
    \end{frame}

    \begin{frame}{Example 3: Will I get an A?}
      \begin{itemize}[<+->]
        \item Let's say I'm taking a class with two midterms (each 25\% weight in final grade) and a final exam (50\% weight in final grade).
        \item I just got my score on the first midterm (75\%).
        \item I want to know how likely it is that I can get 90\% or above on my final grade.
        \item This is hard!
      \end{itemize}
    \end{frame}

    \begin{frame}{Example 3: Will I get an A?}
      Let's start by making some assumptions:
      \begin{itemize}[<+->]
        \item I think I can improve on the second midterm, and then even more on the final.
        \item I'll model my Midterm 2 grade as a normal distribution.
        \item My best guess is that I'll get an 80\% on Midterm 2, and I'm 95\% sure it will be between 70\% and 90\%
        \item So my Midterm 2 grade should be simulated as a normal distribution with mean 80 and SD 5 (since 95\% of a normal distribution is roughly $\pm 2$ SD from the mean).
        \item I think I can improve more on the final; my best guess is that I'll get a 90\%, and I'm 95\% sure I'll get between 80\% and 100\%.
      \end{itemize}
      \lc
    \end{frame}

    \begin{frame}{Example 3: Will I get an A?}
      \begin{itemize}[<+->]
        \item For each run, we will:
          \begin{enumerate}[<+->]
            \item Randomly draw a Midterm 2 score from its normal distribution, and a Final Exam score from its normal distribution.
            \item Calculate a final score for the course, and see if it's over 90\%.
          \end{enumerate}
        \item Then we will count the percentage of runs where we got 90\%+ for the course. That will be our estimate of the probability of getting an A.
      \end{itemize}
    \end{frame}

    \begin{frame}[fragile]
      \fontsize{10}{10}\selectfont
\begin{knitrout}
\definecolor{shadecolor}{rgb}{0.137, 0.137, 0.137}\begin{kframe}
\begin{alltt}
\hlstd{grades} \hlkwb{<-} \hlkwd{replicate}\hlstd{(}\hlnum{10000}\hlstd{, \{}
  \hlstd{midterm1} \hlkwb{<-} \hlnum{75}
  \hlstd{midterm2} \hlkwb{<-} \hlkwd{rnorm}\hlstd{(}\hlnum{1}\hlstd{,} \hlkwc{mean}\hlstd{=}\hlnum{80}\hlstd{,} \hlkwc{sd}\hlstd{=}\hlnum{5}\hlstd{)}
  \hlstd{final.exam} \hlkwb{<-} \hlkwd{rnorm}\hlstd{(}\hlnum{1}\hlstd{,} \hlkwc{mean}\hlstd{=}\hlnum{90}\hlstd{,} \hlkwc{sd}\hlstd{=}\hlnum{5}\hlstd{)}
  \hlnum{0.25}\hlopt{*}\hlstd{midterm1} \hlopt{+} \hlnum{0.25}\hlopt{*}\hlstd{midterm2} \hlopt{+} \hlnum{0.5}\hlopt{*}\hlstd{final.exam}
\hlstd{\})}
\hlkwd{hist}\hlstd{(grades,} \hlkwc{col}\hlstd{=}\hlstr{'orange'}\hlstd{)}
\end{alltt}
\end{kframe}
\input{/tmp/figures/unnamed-chunk-9-1.tikz}

\end{knitrout}
    \end{frame}

    \begin{frame}[fragile]
\begin{knitrout}
\definecolor{shadecolor}{rgb}{0.137, 0.137, 0.137}\begin{kframe}
\begin{alltt}
\hlstd{runs} \hlkwb{<-} \hlkwd{replicate}\hlstd{(}\hlnum{10000}\hlstd{, \{}
  \hlstd{midterm1} \hlkwb{<-} \hlnum{75}
  \hlstd{midterm2} \hlkwb{<-} \hlkwd{rnorm}\hlstd{(}\hlnum{1}\hlstd{,} \hlkwc{mean}\hlstd{=}\hlnum{80}\hlstd{,} \hlkwc{sd}\hlstd{=}\hlnum{5}\hlstd{)}
  \hlstd{final.exam} \hlkwb{<-} \hlkwd{rnorm}\hlstd{(}\hlnum{1}\hlstd{,} \hlkwc{mean}\hlstd{=}\hlnum{90}\hlstd{,} \hlkwc{sd}\hlstd{=}\hlnum{5}\hlstd{)}
  \hlnum{0.25}\hlopt{*}\hlstd{midterm1} \hlopt{+} \hlnum{0.25}\hlopt{*}\hlstd{midterm2} \hlopt{+} \hlnum{0.5}\hlopt{*}\hlstd{final.exam} \hlopt{>=} \hlnum{90}
\hlstd{\})}
\hlkwd{sum}\hlstd{(runs)} \hlopt{/} \hlnum{10000}
\end{alltt}
\begin{verbatim}
[1] 0.0115
\end{verbatim}
\end{kframe}
\end{knitrout}

      \pause
      There's only about a 1.15\% chance that I'll get an A.
    \end{frame}
  \end{darkframes}
\end{document}
