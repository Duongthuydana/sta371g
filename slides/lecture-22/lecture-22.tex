\documentclass{beamer}\usepackage[]{graphicx}\usepackage[]{color}
%% maxwidth is the original width if it is less than linewidth
%% otherwise use linewidth (to make sure the graphics do not exceed the margin)
\makeatletter
\def\maxwidth{ %
  \ifdim\Gin@nat@width>\linewidth
    \linewidth
  \else
    \Gin@nat@width
  \fi
}
\makeatother

\definecolor{fgcolor}{rgb}{1, 0.894, 0.769}
\newcommand{\hlnum}[1]{\textcolor[rgb]{0.824,0.412,0.118}{#1}}%
\newcommand{\hlstr}[1]{\textcolor[rgb]{1,0.894,0.71}{#1}}%
\newcommand{\hlcom}[1]{\textcolor[rgb]{0.824,0.706,0.549}{#1}}%
\newcommand{\hlopt}[1]{\textcolor[rgb]{1,0.894,0.769}{#1}}%
\newcommand{\hlstd}[1]{\textcolor[rgb]{1,0.894,0.769}{#1}}%
\newcommand{\hlkwa}[1]{\textcolor[rgb]{0.941,0.902,0.549}{#1}}%
\newcommand{\hlkwb}[1]{\textcolor[rgb]{0.804,0.776,0.451}{#1}}%
\newcommand{\hlkwc}[1]{\textcolor[rgb]{0.78,0.941,0.545}{#1}}%
\newcommand{\hlkwd}[1]{\textcolor[rgb]{1,0.78,0.769}{#1}}%
\let\hlipl\hlkwb

\usepackage{framed}
\makeatletter
\newenvironment{kframe}{%
 \def\at@end@of@kframe{}%
 \ifinner\ifhmode%
  \def\at@end@of@kframe{\end{minipage}}%
  \begin{minipage}{\columnwidth}%
 \fi\fi%
 \def\FrameCommand##1{\hskip\@totalleftmargin \hskip-\fboxsep
 \colorbox{shadecolor}{##1}\hskip-\fboxsep
     % There is no \\@totalrightmargin, so:
     \hskip-\linewidth \hskip-\@totalleftmargin \hskip\columnwidth}%
 \MakeFramed {\advance\hsize-\width
   \@totalleftmargin\z@ \linewidth\hsize
   \@setminipage}}%
 {\par\unskip\endMakeFramed%
 \at@end@of@kframe}
\makeatother

\definecolor{shadecolor}{rgb}{.97, .97, .97}
\definecolor{messagecolor}{rgb}{0, 0, 0}
\definecolor{warningcolor}{rgb}{1, 0, 1}
\definecolor{errorcolor}{rgb}{1, 0, 0}
\newenvironment{knitrout}{}{} % an empty environment to be redefined in TeX

\usepackage{alltt}
\usepackage{preview}
\usepackage{../371g-slides}
\title{Logistic Regression 1}
\subtitle{Lecture 22}
\author{STA 371G}
\IfFileExists{upquote.sty}{\usepackage{upquote}}{}
\begin{document}
  
  

  \frame{\maketitle}

  % Show outline at beginning of each section
  \AtBeginSection[]{
    \begin{frame}<beamer>
      \tableofcontents[currentsection]
    \end{frame}
  }

  %%%%%%% Slides start here %%%%%%%

  \begin{darkframes}
    \begin{frame}{Announcements}
      \begin{itemize}[<+->]
        \item You'll get Midterm 2 back later this week (a few students still have to take the test).
        \item Get started on your project with your group as soon as you can, so you have plenty of time to iron out any issues.
      \end{itemize}
    \end{frame}

    \begin{frame}{Near, far, wherever you are....}
      \begin{itemize}
        \item How much did the ticket price affect whether someone survived the Titanic?
        \item We have a data set of 1045 passengers on the Titanic; 427 survived
          \begin{itemize}
            \item \textbf{fare}: the amount of money paid for the ticket
            \item \textbf{survived}: a dummy variable indicating whether the passenger surived (1) or not (0)
          \end{itemize}
        \pause
        \item What is unusual about this data set?
      \end{itemize}
    \end{frame}

\begin{frame}[fragile]{What goes wrong with linear regression?}
\begin{knitrout}
\definecolor{shadecolor}{rgb}{0.137, 0.137, 0.137}\begin{kframe}
\begin{alltt}
\hlkwd{plot}\hlstd{(titanic}\hlopt{$}\hlstd{fare, titanic}\hlopt{$}\hlstd{survived)}
\hlstd{model} \hlkwb{<-} \hlkwd{lm}\hlstd{(survived} \hlopt{~} \hlstd{fare,} \hlkwc{data}\hlstd{=titanic)}
\hlkwd{abline}\hlstd{(model)}
\end{alltt}
\end{kframe}
\input{/tmp/figures/unnamed-chunk-2-1.tex}

\end{knitrout}
\end{frame}

    \begin{frame}{The idea behind logistic regression}
      \begin{itemize}
        \item Instead of predicting whether someone survives, let's predict the \emph{probability} that they survive
        \item Let's fit a curve that is always between 0 and 1
      \end{itemize}
    \end{frame}

    \begin{frame}{Odds}
      \begin{itemize}
        \item When something has ``even (1/1) odds,'' the probability of success is $1/2$
        \item When something has ``2/1 odds,'' the probability of success is $2/3$
        \item When something has ``3/2 odds,'' the probability of success is $3/5$
        \item In general, the odds of something happening are $p/(1-p)$
      \end{itemize}
      \lc
    \end{frame}

    \begin{frame}{The logistic regression model}
      Logistic regression models the \alert{log odds} of success $p$ as a linear function of $X$:
      \[
        \log\left(\frac{p}{1-p}\right) = \beta_0 + \beta_1 X + \epsilon
      \]
      This fits an S-shaped curve to the data (we'll see what it looks like later).
    \end{frame}

\begin{frame}[fragile]{Let's try it}
\begin{knitrout}
\definecolor{shadecolor}{rgb}{0.137, 0.137, 0.137}\begin{kframe}
\begin{alltt}
\hlstd{model} \hlkwb{<-} \hlkwd{glm}\hlstd{(survived} \hlopt{~} \hlstd{fare,} \hlkwc{data}\hlstd{=titanic,}
             \hlkwc{family}\hlstd{=binomial)}
\hlkwd{summary}\hlstd{(model)}
\end{alltt}
\end{kframe}
\end{knitrout}
\end{frame}

    \begin{frame}{How to interpret the curve?}
      
      The regression output tells us that our prediction is
      \[
        \log\left(\frac{P(\text{survival})}{1-P(\text{survival})}\right) = -0.794 + 0.012\cdot\text{fare}.
      \]
      \pause
      Let's solve for $P(\text{survival})$:
      \[
        \widehat{P(\text{survival})} = \frac{e^{-0.794 + 0.012\cdot\text{fare}}}{1 + e^{-0.794 + 0.012\cdot\text{fare}}}
      \]
      \lc
    \end{frame}

\begin{frame}[fragile]{Making predictions}
      We can use \texttt{predict} to automate the process of plugging into the equation:
\begin{knitrout}
\definecolor{shadecolor}{rgb}{0.137, 0.137, 0.137}\begin{kframe}
\begin{alltt}
\hlkwd{predict}\hlstd{(model,} \hlkwd{list}\hlstd{(}\hlkwc{fare}\hlstd{=}\hlnum{50}\hlstd{),} \hlkwc{type}\hlstd{=}\hlstr{"response"}\hlstd{)}
\end{alltt}
\begin{verbatim}
    1 
0.453 
\end{verbatim}
\end{kframe}
\end{knitrout}
      
      \[
        \frac{e^{-0.794 + 0.012\cdot 50}}{1 + e^{-0.794 + 0.012\cdot 50}} = 0.453
      \]
\end{frame}

\begin{frame}[fragile]{How to interpret the curve?}
      \[
        \widehat{P(\text{survival})} = \frac{e^{-0.794 + 0.012\cdot\text{fare}}}{1 + e^{-0.794 + 0.012\cdot\text{fare}}}
      \]
\begin{knitrout}
\definecolor{shadecolor}{rgb}{0.137, 0.137, 0.137}
\input{/tmp/figures/unnamed-chunk-7-1.tex}

\end{knitrout}
\end{frame}

    \begin{frame}{Interpreting the coefficients}
      Our prediction equation is:
      \[
        \log\left(\frac{P(\text{survival})}{1-P(\text{survival})}\right) = -0.794 + 0.012\cdot\text{fare}.
      \]
      Let's start with some basic, but not particularly useful, interpretations:
      \begin{itemize}[<+->]
        \item When $\text{fare}=0$, we predict that the log odds will be $-0.794$ \pause, so the probability of survival is predicted to be 31\%.
        \item When fare increases by \pounds 1, we predict that the log odds will increase by $0.012$.
      \end{itemize}
      \note{LC question in the middle}
    \end{frame}

    \begin{frame}{Interpreting the coefficients}
      Let's rewrite the prediction equation as:
      \[
        \text{Predicted odds of survival} = e^{-0.794 + 0.012\cdot\text{fare}}
      \]
      Increasing the fare by \pounds 1 will \emph{multiply} the odds by $e^{0.012}=1.012$; i.e., increase the odds by $1.2$\%.

      \bigskip\pause
      Increasing the fare by \pounds 10 will \emph{multiply} the odds by $e^{10 \cdot 0.012}=1.129$; i.e., increase the odds by $12.9$\%.
    \end{frame}

    \begin{frame}{Testing the null hypothesis}
      \begin{center}
        As in regular linear regression, the overall null hypothesis is that $\beta_1=0$; we can test this by using the $p$-value for that variable on the output.

        \bigskip\pause
        Since $p$ is very small, we can reject the null hypothesis that $\beta_1=0$; i.e., there is a statistically significant relationship between fare and survival.
      \end{center}
    \end{frame}

    \begin{frame}{How good is our model?}
      \begin{itemize}[<+->]
        \item Unfortunately, the typical $R^2$ metric isn't available for logistic regression.
        \item However, there are many ``pseudo-$R^2$'' metrics that indicate model fit.
      \end{itemize}
    \end{frame}

\begin{frame}[fragile]{Take 1: How many cases did we accurately predict?}
      We could use our model to make a prediction of survival (or not), based on the probability. Suppose we say that our prediction is:
      \[
        \text{Prediction} = \begin{cases}
          \text{survival}, & \text{if $\widehat{P(\text{survival})} \geq 0.5$}, \\
          \text{tragedy}, & \text{if $\widehat{P(\text{survival})} < 0.5$}. \\
        \end{cases}
      \]

      \pause
      Now we can compute the fraction of people whose survival we correctly predicted:

      \fontsize{9}{9}\selectfont
\begin{knitrout}
\definecolor{shadecolor}{rgb}{0.137, 0.137, 0.137}\begin{kframe}
\begin{alltt}
\hlstd{predicted.survival} \hlkwb{<-} \hlstd{(}\hlkwd{predict}\hlstd{(model,} \hlkwc{type}\hlstd{=}\hlstr{'response'}\hlstd{)} \hlopt{>=} \hlnum{0.5}\hlstd{)}
\hlstd{actual.survival} \hlkwb{<-} \hlstd{(titanic}\hlopt{$}\hlstd{survived} \hlopt{==} \hlnum{1}\hlstd{)}
\hlkwd{sum}\hlstd{(predicted.survival} \hlopt{==} \hlstd{actual.survival)} \hlopt{/} \hlkwd{nrow}\hlstd{(titanic)}
\end{alltt}
\begin{verbatim}
[1] 0.644
\end{verbatim}
\end{kframe}
\end{knitrout}
      \note{LC question in middle}
\end{frame}

\begin{frame}[fragile]{Take 1: How many cases did we accurately predict?}
      64.4\% sounds pretty good---what should we compare it against?

      \bigskip
      \pause

      We should compare 64.4\% against what we would have gotten if we just predicted the most common outcome (not surviving) for everyone, without using any other information:
      \pause

\begin{knitrout}
\definecolor{shadecolor}{rgb}{0.137, 0.137, 0.137}\begin{kframe}
\begin{alltt}
\hlnum{1} \hlopt{-} \hlkwd{sum}\hlstd{(actual.survival)} \hlopt{/} \hlkwd{nrow}\hlstd{(titanic)}
\end{alltt}
\begin{verbatim}
[1] 0.591
\end{verbatim}
\end{kframe}
\end{knitrout}
\end{frame}

    \begin{frame}{Take 2: McFadden's pseudo-$R^2$}
      To get a metric on the usual 0-1 scale (like regular $R^2$), McFadden's pseudo-$R^2$ can be used (this is what is described in the reading):
      \[
        \text{pseudo-}R^2 = 1 - \frac{\text{residual deviance}}{\text{null deviance}} = 1 - \frac{1339.941}{1413.571} = 0.052
      \]
    \end{frame}

\begin{frame}[fragile]
      \fontsize{9}{9}\selectfont
\begin{knitrout}
\definecolor{shadecolor}{rgb}{0.137, 0.137, 0.137}\begin{kframe}
\begin{verbatim}

Call:
glm(formula = survived ~ fare, family = binomial, data = titanic)

Deviance Residuals: 
   Min      1Q  Median      3Q     Max  
-2.231  -0.928  -0.898   1.335   1.528  

Coefficients:
            Estimate Std. Error z value Pr(>|z|)    
(Intercept)  -0.7941     0.0844   -9.41  < 2e-16 ***
fare          0.0121     0.0017    7.13  9.9e-13 ***
---
Signif. codes:  0 '***' 0.001 '**' 0.01 '*' 0.05 '.' 0.1 ' ' 1

(Dispersion parameter for binomial family taken to be 1)

    Null deviance: 1413.6  on 1044  degrees of freedom
Residual deviance: 1339.9  on 1043  degrees of freedom
AIC: 1344

Number of Fisher Scoring iterations: 4
\end{verbatim}
\end{kframe}
\end{knitrout}
      \note{Do extra LC practice questions if time.}
\end{frame}

    \begin{frame}{What else can we use logistic regression for?}
      \begin{itemize}
        \item \textbf{Finance:} Predicting which customers are most likely to default on a loan
        \item \textbf{Advertising:} Predicting when a customer will respond positively to an advertising campaign
        \item \textbf{Marketing:} Predicting when a customer will purchase a product or sign up for a service
      \end{itemize}
    \end{frame}
  \end{darkframes}

\end{document}
