\documentclass{beamer}\usepackage[]{graphicx}\usepackage[]{color}
%% maxwidth is the original width if it is less than linewidth
%% otherwise use linewidth (to make sure the graphics do not exceed the margin)
\makeatletter
\def\maxwidth{ %
  \ifdim\Gin@nat@width>\linewidth
    \linewidth
  \else
    \Gin@nat@width
  \fi
}
\makeatother

\definecolor{fgcolor}{rgb}{1, 0.894, 0.769}
\newcommand{\hlnum}[1]{\textcolor[rgb]{0.824,0.412,0.118}{#1}}%
\newcommand{\hlstr}[1]{\textcolor[rgb]{1,0.894,0.71}{#1}}%
\newcommand{\hlcom}[1]{\textcolor[rgb]{0.824,0.706,0.549}{#1}}%
\newcommand{\hlopt}[1]{\textcolor[rgb]{1,0.894,0.769}{#1}}%
\newcommand{\hlstd}[1]{\textcolor[rgb]{1,0.894,0.769}{#1}}%
\newcommand{\hlkwa}[1]{\textcolor[rgb]{0.941,0.902,0.549}{#1}}%
\newcommand{\hlkwb}[1]{\textcolor[rgb]{0.804,0.776,0.451}{#1}}%
\newcommand{\hlkwc}[1]{\textcolor[rgb]{0.78,0.941,0.545}{#1}}%
\newcommand{\hlkwd}[1]{\textcolor[rgb]{1,0.78,0.769}{#1}}%
\let\hlipl\hlkwb

\usepackage{framed}
\makeatletter
\newenvironment{kframe}{%
 \def\at@end@of@kframe{}%
 \ifinner\ifhmode%
  \def\at@end@of@kframe{\end{minipage}}%
  \begin{minipage}{\columnwidth}%
 \fi\fi%
 \def\FrameCommand##1{\hskip\@totalleftmargin \hskip-\fboxsep
 \colorbox{shadecolor}{##1}\hskip-\fboxsep
     % There is no \\@totalrightmargin, so:
     \hskip-\linewidth \hskip-\@totalleftmargin \hskip\columnwidth}%
 \MakeFramed {\advance\hsize-\width
   \@totalleftmargin\z@ \linewidth\hsize
   \@setminipage}}%
 {\par\unskip\endMakeFramed%
 \at@end@of@kframe}
\makeatother

\definecolor{shadecolor}{rgb}{.97, .97, .97}
\definecolor{messagecolor}{rgb}{0, 0, 0}
\definecolor{warningcolor}{rgb}{1, 0, 1}
\definecolor{errorcolor}{rgb}{1, 0, 0}
\newenvironment{knitrout}{}{} % an empty environment to be redefined in TeX

\usepackage{alltt}
\usepackage{../371g-slides}
\title{Midterm 2 review}
\subtitle{Lecture 21}
\author{STA 371G}
\IfFileExists{upquote.sty}{\usepackage{upquote}}{}
\begin{document}
  
  

  \frame{\maketitle}

  % Show outline at beginning of each section
  \AtBeginSection[]{
    \begin{frame}<beamer>
      \tableofcontents[currentsection]
    \end{frame}
  }

  %%%%%%% Slides start here %%%%%%%

  \begin{darkframes}
    \begin{frame}
      \frametitle{Confidence vs Prediction}
      \begin{itemize}
      \item \textbf{Confidence} interval is a range where you are confident the true parameter value is...
        \begin{itemize}
        \item 95\% confidence interval for a slope is a range where you're 95\% sure the slope would be if you saw all possible data points.
        \item 95\% confidence interval for $\hat{Y}_i$ is a range where you're 95\% sure the regression line would be.
        \item Remember, the regression line gives the expected value, so if you want a range where you are 95\% sure the average for a given $X$ is... that's the confidence interval.
        \end{itemize}
      \item \textbf{Prediction} interval is a range where you predict an indivicual data point would lie.
        \begin{itemize}
        \item If we're predicting salary vs age, the salary of an individual 21 year old is not going to lie exactly on the line.
        \item Prediction intervals also take into account the residual variance.
        \end{itemize}
      \end{itemize}
    \end{frame}


    \begin{frame}
      \frametitle{AIC, BIC and adjusted $R^2$}
      \begin{itemize}
      \item We know $R^2$ will always go up as we add more predictors...
        \item But, our regression might not actually be better... we are likely overfitting.
        \item AIC, BIC and adjusted $R^2$ are all alternative measures that we can use to tell ``how good'' our regression is.
        \item They take into account both how well the model predicts, and how complicated they are (how many predictors)... if two models predict equally well, they will pick the simpler model.
        \item They all weight prediction and parsimony differently... for this course we're not going to look at the details.
        \item Checking several can give you a small set of candidate models to choose between.
      \end{itemize}
    \end{frame}

    \begin{frame}
      \frametitle{Multicollinearity, correlation and VIF}
      \begin{itemize}
      \item Multicollinearity occurs when two variables are correlated, and the part that isn't correlated doesn't tell us much.
      \item It doesn't mean that the two variables are bad... just that the correlation is making it hard to identify their individual effects.
      \item We can usually improve our model by removing one of the offending variables.
        \pause
      \item Looking at the pairwise correlations can help us spot multicollinearity -- but we can have multicollinearity at middling correlation levels.
      \item VIF is a way of looking at how uncertain we are about the effect of a variable in our full model, vs in a model with that variable alone.
      \item VIF $>$ 5 indicates multicollinearity.
      \item We don't necessarily remove \textit{all} variables with VIF $>$ 5... try removing each one individually to see if it clears up the multicollinearity.
      \end{itemize}
    \end{frame}
  \end{darkframes}
\end{document}
