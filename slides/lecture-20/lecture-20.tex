\documentclass{beamer}\usepackage[]{graphicx}\usepackage[]{color}
%% maxwidth is the original width if it is less than linewidth
%% otherwise use linewidth (to make sure the graphics do not exceed the margin)
\makeatletter
\def\maxwidth{ %
  \ifdim\Gin@nat@width>\linewidth
    \linewidth
  \else
    \Gin@nat@width
  \fi
}
\makeatother

\definecolor{fgcolor}{rgb}{1, 0.894, 0.769}
\newcommand{\hlnum}[1]{\textcolor[rgb]{0.824,0.412,0.118}{#1}}%
\newcommand{\hlstr}[1]{\textcolor[rgb]{1,0.894,0.71}{#1}}%
\newcommand{\hlcom}[1]{\textcolor[rgb]{0.824,0.706,0.549}{#1}}%
\newcommand{\hlopt}[1]{\textcolor[rgb]{1,0.894,0.769}{#1}}%
\newcommand{\hlstd}[1]{\textcolor[rgb]{1,0.894,0.769}{#1}}%
\newcommand{\hlkwa}[1]{\textcolor[rgb]{0.941,0.902,0.549}{#1}}%
\newcommand{\hlkwb}[1]{\textcolor[rgb]{0.804,0.776,0.451}{#1}}%
\newcommand{\hlkwc}[1]{\textcolor[rgb]{0.78,0.941,0.545}{#1}}%
\newcommand{\hlkwd}[1]{\textcolor[rgb]{1,0.78,0.769}{#1}}%
\let\hlipl\hlkwb

\usepackage{framed}
\makeatletter
\newenvironment{kframe}{%
 \def\at@end@of@kframe{}%
 \ifinner\ifhmode%
  \def\at@end@of@kframe{\end{minipage}}%
  \begin{minipage}{\columnwidth}%
 \fi\fi%
 \def\FrameCommand##1{\hskip\@totalleftmargin \hskip-\fboxsep
 \colorbox{shadecolor}{##1}\hskip-\fboxsep
     % There is no \\@totalrightmargin, so:
     \hskip-\linewidth \hskip-\@totalleftmargin \hskip\columnwidth}%
 \MakeFramed {\advance\hsize-\width
   \@totalleftmargin\z@ \linewidth\hsize
   \@setminipage}}%
 {\par\unskip\endMakeFramed%
 \at@end@of@kframe}
\makeatother

\definecolor{shadecolor}{rgb}{.97, .97, .97}
\definecolor{messagecolor}{rgb}{0, 0, 0}
\definecolor{warningcolor}{rgb}{1, 0, 1}
\definecolor{errorcolor}{rgb}{1, 0, 0}
\newenvironment{knitrout}{}{} % an empty environment to be redefined in TeX

\usepackage{alltt}
\usepackage{../371g-slides}
\title{Regression Lab}
\subtitle{Lecture 20}
\author{STA 371G}
\IfFileExists{upquote.sty}{\usepackage{upquote}}{}
\begin{document}
  
  

  \frame{\maketitle}

  % Show outline at beginning of each section
  \AtBeginSection[]{
    \begin{frame}<beamer>
      \tableofcontents[currentsection]
    \end{frame}
  }

  %%%%%%% Slides start here %%%%%%%

  \begin{darkframes}
    \begin{frame}
      \begin{enumerate}
        \item Sit with the other members of your project group---today you'll work with them on a model-building workshop.
        \item Log in to Learning Catalytics and click on the Model Building Workshop module.
        \item Everyone in your group should enter your group name and number as the team name, e.g., ``Group 15''.
      \end{enumerate}
    \end{frame}

    \begin{frame}{Reminders}
      \begin{itemize}
        \item Midterm 2 is this Thursday/Friday
        \item Make sure to sign up for a time slot if you haven't already
        \item You can bring 2 pages of notes to the test
      \end{itemize}
    \end{frame}

    \begin{frame}{Model building workshop}
      \begin{itemize}
        \item Today you'll work through a set of interesting questions about predicting income using data from the General Social Survey (GSS)
        \item The GSS is an annual survey of attitudes and behaviors that has been conducted since the 1970s
        \item Much of the work is in getting the data into a form that you can work with and dealing with data issues---this will be good practice for your project!
      \end{itemize}
    \end{frame}

    \begin{frame}{Loading the data}
      \begin{itemize}
        \item Load the GSS data into a variable
        \item Rename the column to simpler names that are easy to remember and type
        \item \alert{Very important:} save all of your cleaning steps into a script so you can easily reproduce them later!
      \end{itemize}
    \end{frame}

    \begin{frame}{Cleaning data pass 1}
      \begin{itemize}
        \item Replace missing data codes with \texttt{NA}
        \item Convert numbers that were read in as character fields (strings) back to numbers
        \item Transform values to something more meaningful where appropriate
      \end{itemize}
    \end{frame}

    \begin{frame}{Model building pass 1}
      \begin{itemize}
        \item Select a meaningful subset of the data
        \item Conduct initial model building
      \end{itemize}
    \end{frame}

    \begin{frame}{Cleaning data pass 2}
      \begin{itemize}
        \item Use boxplots to think about which levels are meaningful
        \item Transform character fields (strings) to levels
      \end{itemize}
    \end{frame}

    \begin{frame}{Model building pass 2}
      \begin{itemize}
        \item Collapse levels where appropriate
        \item Build another interesting model
      \end{itemize}
    \end{frame}

    \begin{frame}{Model selection}
      \begin{itemize}
        \item Use automated tools (stepwise and best-subsets regression) to help narrow down the possible models
        \item Examine and remove incomplete rows
        \item Use the output of automated tools to decide what to explore further
      \end{itemize}
    \end{frame}
  \end{darkframes}
\end{document}
